\documentclass{article}
\usepackage{enumitem}

\newlist{FR}{enumerate}{1}
\setlist[FR]{label=FR-\arabic*:}


\renewcommand{\labelenumii}{\theenumii}
\renewcommand{\theenumii}{\theenumi.\arabic{enumii}.}

\usepackage{graphicx}
\graphicspath{{Figures/}}


\title{Software Requirements Specification}
\date{24 February 2017}
\author{Team Dodger}

\begin{document}
	\pagenumbering{gobble}
	\maketitle
	\newpage
	\tableofcontents
	\newpage
	\pagenumbering{arabic}
	
	\section{Introduction}
	\subsection{Purpose}
	\subsection{Scope}
	\subsection{Definitions, Acronyms and Abbreviations}
	\subsection{References}
	\subsection{Overview}
	
	\section{Overall Description}
	\subsection{Product Perspective}
	The main system will be a server on campus that is connected to a database where user's details are stored such as degree, interests and timetable. The user will log into the main system by means of a user friendly GUI provided on the mobile application that sends the user's details to the server over the internet to look at the authenticity and correctness of the provided details. Thereafter the main system will load information about the user and populate the GUI with the relevant information based on the category of the current user be it a student or administrator. The application will then use wifi connections, cell phone towers and GPS to determine the user's position and then triangulate them to the destination class or activity based on these calculations. When administrators update the application or database through the server or the application itself it will notify the user and update the relevant information for the user, for example if a class is cancelled, it will update the user's timetable and notify the user of this update through the GUI as well as push notifications. The map of campus can be loaded onto the mobile application to reduce long term internet usage and only the destination and route to get there will be sent over the network.
	\subsection{Product Function}
	\begin{itemize}
	\item Navigation from and to a location
	\item Store users timetable to automatically tell user where to go next
	\item View more information about a specific location
	\item See upcoming events when at a location or based on users interests
	\item Reward user for completing challenges 
	\item Update user timetables if classes are cancelled or if there will be a test
	\end{itemize}
	\subsection{User Characteristics}
	Four categories of users will be present:
	\begin{itemize}
	\item Guest user: Basic education level is needed as this user will just be making use of the navigation system and search system so a basic technical skill will be needed.
	\item Main user (student): This user has at least a high school level education and will be using more advanced parts of the system such as setting up a class timetable, looking at and competing in the reward system and managing their profile.
	\item Administrator / lecturer: This user has at least a high school level education and will need a more advanced technical knowledge as this user will be setting up events, cancelling classes and fixing or updating the system for other users.  
	\item 3rd party Rewards manager: High school level education will be needed as more advanced tasks will be left up to this user such as set up challenges for the other users and include rewards if challenge is completed. An advanced technical knowledge will be needed by this user.

	\end{itemize}
	\subsection{Constraints}
	This section describes restrictions on the options that are available when developing the application within feasable regions
		\begin{itemize}
			\item Connections are limited to different types of networks at different locations. GPS cannot be used within buildings and some buildings lack a strong Wi-Fi signal. Mobile networks may also switch to EDGE in some buildings where faster connections aren't available.
			\item Application is initially constrained to Android and iOS only.
			\item Application is designed for approximately 30000 users at any given time which can be seen as a constraint on the number of active users the system can handle.
			\item The application can experience lengthy response times given that there is a constraint resulting from the capacity of the databases.\newline
		\end{itemize}
	
	\subsection{Assumptions and Dependencies}
	
	\section{Specific Requirements}
	This section expands on the functional requirements of the system. It gives a detailed 	description of the system and all of its use cases.
	
	\subsection{External Interface Requirements}
	Not relevant
	\subsection{Functional Requirements}
	This section includes all functional requirements in detail. It includes all use case diagrams, Actor-System interaction diagrams as well as a traceability matrix.	
	
	\subsubsection{Use cases}
	\begin{enumerate}
		\item \underline{Navigation Subsystem}
			
		
	\begin{enumerate}
		\item Get current location
		\begin{enumerate}
			\item \textbf{Description:} The NavUP system must be able to determine a user’s location at any point in time while the user is on the Hatfield campus. The location must be determined regardless of whether the user is indoors or outdoors.
			\item \textbf{Precondition:} The user must have an active account and must be within range of WiFi routers.
			\item \textbf{Postcondition:} The user’s location is determined and displayed.\newline
		\end{enumerate}
		
		\item Search location
		\begin{enumerate}
			\item \textbf{Description:} The NavUP system must provide functionality that enables a user to search for any location (lecture hall, day-house, restaurant) on the Hatfield Campus.
			\item \textbf{Precondition:} The user must have an active account
			\item \textbf{Postcondition:} Matching locations are returned to the user. If no buildings match the search criteria, an appropriate error message is displayed.\newline
		\end{enumerate}
		
		
		\item View location details
		\begin{enumerate}
			\item \textbf{Description:} The NavUP system must allow users to view details related to specific locations. This could include restaurant menus, lecture hall timetable schedules as well as images of the buildings.
			\item \textbf{Precondition:} The user must have an active account and a valid location must be selected on the map.
			\item \textbf{Postcondition:} Relevant location details shown to user.\newline
		\end{enumerate}
		
		\item View places of interest
		\begin{enumerate}
			\item \textbf{Description:} The NavUP system must be able to display places of interests to a user based on their current location. This will include places like restaurants and day-houses that must be displayed in a list form. 
			\item \textbf{Precondition:} The user must have an active account and their current location must be known.  
			\item \textbf{Postcondition:} Relevant places of interest are listed and displayed to the user based on their location.\newline
		\end{enumerate}
		
		\item Navigate to location
		\begin{enumerate}
			\item \textbf{Description:} The NavUP system must be able to provide directions and navigate to a location given the user’s current location as well as a desired destination. The system should calculate the most optimal route by looking at the shortest path as well as pedestrian traffic.
			\item \textbf{Precondition:} The user must have an active account. The user’s current location must be known and the must have specified a destination through the search interface.
			\item \textbf{Postcondition:} The user is provided with directions from their current location to their desired destination.\newline
		\end{enumerate}
		
		\item Show pedestrian traffic
		\begin{enumerate}
			\item \textbf{Description:} The NavUP system must be able to display pedestrian traffic on campus in the form of a heatmap. When navigating to a specified location, the system must show traffic on that specific route. A user should also be able to view an overall heatmap of the campus to see traffic.
			\item \textbf{Precondition:} Users must all have the NavUP app installed and must be registered in order for them to show up on the heatmap.
			\item \textbf{Postcondition:} A heatmap of the campus is displayed. 
		\end{enumerate}
	\end{enumerate}
	\begin{figure}[h!]
		\includegraphics[scale=0.5]{Navigation_Subsystem.png}
		\caption{Navigation Subsystem}	
	\end{figure}
	
	
	\item \underline{Location Management Subsystem}
	\item \underline{User Account Management Subsystem}
	
	
	\item \underline{Entertainment Subsystem}
			\begin{enumerate}
		\item View events
		\begin{enumerate}
			\item \textbf{Description:} The NavUP system must enable users to view all events that are happening around campus in chronological order. The system should suggest events to a user based on their preferences and most visited locations.
			\item \textbf{Precondition:} The user must have an active account and must be logged in.
			\item \textbf{Postcondition:} Various campus-wide events are returned to the user.\newline
		\end{enumerate}
		
		\item Save event
		\begin{enumerate}
			\item \textbf{Description:} The NavUP system must enable users to save events that they are interested so that they can be viewed later.
			\item \textbf{Precondition:} The user must have an active account, must be logged in and there must be events available to save.
			\item \textbf{Postcondition:} An event is saved.\newline
		\end{enumerate}
		
		\item Delete event
		\begin{enumerate}
			\item \textbf{Description:} The NavUP system must enable a user to delete any saved events
			\item \textbf{Precondition:} The user must have an active account, must be logged in and must have saved events
			\item \textbf{Postcondition:} A saved event is deleted .\newline
		\end{enumerate}
	\end{enumerate}
	\begin{figure}[h!]
		\includegraphics[scale=0.5]{Entertainment_Subsystem.png}
		\caption{Navigation Subsystem}	
	\end{figure}	
	
	\item \underline{Achievements Subsystem}
	\item \underline{Administration Subsystem}
	\end{enumerate}
	
	
	\subsection{Performance Requirements}
	Not relevant
	
	\subsection{Design Constraints}
	This section describes restrictions on design alternatives regarding standards and limitations of hardware capabilities
	\begin{enumerate}
		\item \textbf{Storage space}\newline
			 \textbf{Description:} The amount of storage space required by the application must be within the maximum storage limits of a budget phone to accomodate a range of phones typically used by students.
			\begin{itemize}
			\item \textbf{Maximum:} 90MB
			\item \textbf{Reasonable:} 40MB
			\item \textbf{Optimal:} 10MB.\newline
		\end{itemize}
	
		\item \textbf{Memory usage}\newline
			\textbf{Description:} The amount of RAM used by the application should be a reasonable amount considering that some smartphones only have a capacity of 1GB RAM
			\begin{itemize}
			\item \textbf{Maximum:} 150MB
			\item \textbf{Reasonable:} 90MB
			\item \textbf{Optimal:} 40MB.\newline
		\end{itemize}
	\end{enumerate}
	
	\subsection{Software System Attributes}
	Not relevant
	\subsection{Other Requirements}
	Not relevant

\end{document}\grid
\grid
